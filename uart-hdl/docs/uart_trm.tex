\documentclass[a4paper,12pt]{article}
\usepackage[margin=1in]{geometry}
\usepackage{longtable}
\usepackage{array}
\usepackage{titlesec}
\usepackage{fancyhdr}
\usepackage{color}
\usepackage{xcolor}
\usepackage{listings}
\usepackage{booktabs}
\usepackage[urlcolor=blue]{hyperref}
\usepackage{fontawesome5}

\pagestyle{fancy}
\fancyhf{}
\rhead{UART Register TRM}
\lhead{}
\rfoot{\thepage}

\titleformat{\section}{\normalfont\Large\bfseries}{\thesection}{1em}{}

\begin{document}

\begin{titlepage}
    \centering
    \vspace*{2cm}
    {\Huge \bfseries UART HDL Technical Reference Manual \par}
    \vspace{2cm}
    {\Large Version 1.0 \par}
    \vspace{1cm}
    {\large \today \par}
    \vfill
    {\large Gregory Williams}
\end{titlepage}

\tableofcontents
\newpage

\section{Device Operation}
You can find the HDL for this module in this github repository:
\href{https://github.com/GregWills97/fpga-uart-project}
{\textcolor{blue}{GregWills97/fpga-uart-project}} \faGithub

\section{Configuration Registers}
\subsection{Register Overview}
This section describes the hardware configuration registers for the UART HDL module.

\subsection{Data Register: UARTDR}
The UARTDR is the data register for the UART device.

\noindent
\\To transmit words, the user issues a write to this register. When written to, the data is placed
into the transmit FIFO and the write address is incremented. Data is automatically prefixed with a
start bit and appended with parity and stop bits if needed.

\noindent
\\Received byte reads are performed by issuing a read to the UARTDR register. The read will return data
from the receive FIFO and increment the FIFO address.

\subsubsection*{Field Description}
\begin{longtable}{|c|c|c|c|m{7cm}|}
\hline
\textbf{Bits} & \textbf{Name} & \textbf{Access} & \textbf{Reset} & \textbf{Description} \\
\hline
31:8 & RESERVED & - & 0x0 & Reserved. \\
\hline
7:0 & DATA & R/W & 0x00 & Receive (read) data\par Transmit (write) data \\
\hline
\end{longtable}

\vspace{1em}
\noindent\textbf{Notes:}
\begin{itemize}
    \item Writes will be ignored if transmit FIFO is full.
\end{itemize}

\end{document}
