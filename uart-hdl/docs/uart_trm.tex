\documentclass[a4paper,12pt]{article}
\usepackage[margin=1in]{geometry}
\usepackage{longtable}
\usepackage{array}
\usepackage{titlesec}
\usepackage{fancyhdr}
\usepackage{color}
\usepackage{xcolor}
\usepackage{listings}
\usepackage{booktabs}
\usepackage[urlcolor=blue]{hyperref}
\usepackage{fontawesome5}
\usepackage{float}
\usepackage{chngcntr}

\pagestyle{fancy}
\fancyhf{}
\rhead{UART Register TRM}
\lhead{}
\rfoot{\thepage}
\counterwithin{table}{section}
\titleformat{\section}{\normalfont\Large\bfseries}{\thesection}{1em}{}

\begin{document}

\begin{titlepage}
    \centering
    \vspace*{2cm}
    {\Huge \bfseries UART HDL Technical Reference Manual \par}
    \vspace{2cm}
    {\Large Version 1.0 \par}
    \vspace{1cm}
    {\large \today \par}
    \vfill
    {\large Gregory Williams}
\end{titlepage}

\tableofcontents
\newpage

\section{Device Operation}
You can find the HDL for this module in this github repository:
\href{https://github.com/GregWills97/fpga-uart-project}
{\textcolor{blue}{GregWills97/fpga-uart-project}} \faGithub

\section{Configuration Registers}
\subsection{Register Overview}
This section describes the hardware configuration registers for the UART HDL module.

% UARTDR REGISTER
\subsection{Data Register: UARTDR}
The UARTDR is the data register for the UART device.

\noindent\\
To transmit words, the user issues a write to this register. When written to, the data is placed
into the transmit FIFO and the write address is incremented. Data is automatically prefixed with a
start bit and appended with parity and stop bits if needed.

\noindent\\
Received byte reads are performed by issuing a read to the UARTDR register. The read will return
data from the receive FIFO and increment the FIFO address along with any errors associated with that
character. A write to bits 7:0 of this register clears the error bits.

\noindent\\
The receiver detects the following errors:
\begin{itemize}
\item\textbf{Overrun Error}: when HIGH indicates that the rx fifo is full and data was received.

\item\textbf{Break Error}: when HIGH indicates that the rx input has been held LOW for longer than a
full-word transmission.

\item\textbf{Parity Error}: when HIGH indicates that the parity of the data character does not
match the parity indicated by the parity setting in the LCTRL register.

\item\textbf{Framing Error}: when HIGH indicates invalid stop bit.
\end{itemize}

\subsubsection*{Field Description}
\renewcommand{\arraystretch}{1.5}
\begin{table}[H]
\centering
\begin{tabular}{|c|c|c|c|m{8cm}|}
\hline
\textbf{Bits} & \textbf{Name} & \textbf{Access} & \textbf{Reset} & \textbf{Description} \\
\hline
31:12 & RESERVED & - & 0x0 & Reserved. \\
\hline
11 & OE & R & 0x0 & Overrun error bit, set HIGH when rx fifo is full.\\
\hline
10 & BE & R & 0x0 & Break error bit, set HIGH when break error detected.\\
\hline
9 & PE & R & 0x0 & Parity error bit, set HIGH when parity error detected.\\
\hline
8 & FE & R & 0x0 & Framing error bit, set HIGH when framing error detected.\\
\hline
7:0 & DATA & R/W & 0x00 &
READ: data from rx fifo is returned.\par
WRITE: data written is put on the tx line.\\
\hline
\end{tabular}
\caption{UARTDR Field Description}\label{tab:UARTDR}
\end{table}

\vspace{1em}
\noindent\textbf{Notes:}
\begin{itemize}
    \item Writes will be ignored if transmit FIFO is full.
\end{itemize}

\end{document}
